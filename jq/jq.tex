\documentclass{beamer}

\usefonttheme{serif}
\usepackage{charter}

\usepackage[utf8]{inputenc}
\usepackage[english]{babel}
\usepackage{listings}
\usepackage{graphicx}
 \usepackage{epstopdf}

\setbeamertemplate{navigation symbols}{}
\setbeamertemplate{footline}[frame number]{}
%\setbeamertemplate{footline}{\hfill\insertframenumber~\vrule~\inserttotalframenumber}
\lstset{
  stepnumber=1,
  breaklines=true,
  basicstyle=\ttfamily\scriptsize,
  numberstyle=\tiny,
  commentstyle=\color{gray},
  showstringspaces=false,
  keepspaces=true,
  escapeinside=\#\#
}

\renewcommand\big[1]{
  \begin{center}
    \Large{#1}
  \end{center}
}

\begin{document}

\begin{frame}
  \centering\Huge{jq}
\end{frame}

\begin{frame}
  \centering\Huge{Principles}
\end{frame}

\begin{frame}[fragile]
  \big{Basic Principle \#1}

  \begin{center}
    jq reads JSON values from stdin, applies a \textit{filter} to them
  \end{center}

  \begin{lstlisting}
    > jq FILTER < values.json
  \end{lstlisting}
\end{frame}

\begin{frame}
  \big{Basic Principle \#2}

  \begin{center}
    A filter consumes $m$ JSON values, produces $n$ JSON values
  \end{center}

  \begin{center}
    \includegraphics[scale=0.2]{jq-filter.pdf}
  \end{center}
\end{frame}

\begin{frame}[fragile]
  \big{Basic Principle \#3}

  \begin{center}
    Filters can be piped together
  \end{center}

  \begin{lstlisting}
    > jq 'FILTER1 | FILTER2 | FILTER3' < values.json
  \end{lstlisting}
\end{frame}

\begin{frame}
  \centering\Huge{Simple Tasks}
\end{frame}


\begin{frame}
  \centering\Huge{Miscellany}
\end{frame}

\begin{frame}
  \big{Some Tips}

  \begin{itemize}
    \item jq is slow; use \textit{grep} to reduce the amount of data sent to jq
    \item if a filter is too long, write it in a file and use \textit{jq -f FILE}
  \end{itemize}
\end{frame}

\end{document}
