\documentclass{beamer}

\usefonttheme{serif}
\usepackage{charter}

\usepackage[utf8]{inputenc}
\usepackage[english]{babel}
\usepackage{listings}
\usepackage{graphicx}
 \usepackage{epstopdf}

\setbeamertemplate{navigation symbols}{}
\setbeamertemplate{footline}[frame number]{}
%\setbeamertemplate{footline}{\hfill\insertframenumber~\vrule~\inserttotalframenumber}
\lstset{
  stepnumber=1,
  breaklines=true,
  basicstyle=\ttfamily\scriptsize,
  numberstyle=\tiny,
  commentstyle=\color{gray},
  showstringspaces=false,
  keepspaces=true,
  escapeinside=\#\#
}

\renewcommand\big[1]{
  \begin{center}
    \Large{#1}
  \end{center}
}

\begin{document}

\begin{frame}
  \centering\Huge{GNU Parallel}
\end{frame}

\begin{frame}[fragile]
    \big{The Extortion Scheme}
\tiny
\begin{verbatim}
~ :) parallel --citation
Academic tradition requires you to cite works you base your article on.
When using programs that use GNU Parallel to process data for publication
please cite:

@article{Tange2011a,
  title = {GNU Parallel - The Command-Line Power Tool},
  author = {O. Tange},
  address = {Frederiksberg, Denmark},
  journal = {;login: The USENIX Magazine},
  month = {Feb},
  number = {1},
  volume = {36},
  url = {http://www.gnu.org/s/parallel},
  year = {2011},
  pages = {42-47},
  doi = {http://dx.doi.org/10.5281/zenodo.16303}
}

(Feel free to use \nocite{Tange2011a})

This helps funding further development; AND IT WON'T COST YOU A CENT.
If you pay 10000 EUR you should feel free to use GNU Parallel without citing.

If you send a copy of your published article to tange@gnu.org, it will be
mentioned in the release notes of next version of GNU Parallel.
\end{verbatim}
\end{frame}

\end{document}
