\documentclass{beamer}

\usefonttheme{serif}
\usepackage{charter}

\usepackage[utf8]{inputenc}
\usepackage[english]{babel}
\usepackage{listings}
\usepackage{graphicx}
 \usepackage{epstopdf}

\setbeamertemplate{navigation symbols}{}
\setbeamertemplate{footline}[frame number]{}
%\setbeamertemplate{footline}{\hfill\insertframenumber~\vrule~\inserttotalframenumber}
\lstset{
  stepnumber=1,
  breaklines=true,
  basicstyle=\ttfamily\scriptsize,
  numberstyle=\tiny,
  commentstyle=\color{gray},
  showstringspaces=false,
  keepspaces=true,
  escapeinside=\#\#
}

\renewcommand\big[1]{
  \begin{center}
    \Large{#1}
  \end{center}
}

\begin{document}

\begin{frame}
  \centering\Huge{GNU Parallel}
\end{frame}

\begin{frame}[fragile]
    \big{The Extortion Scheme}
\tiny
\begin{verbatim}
~ :) parallel --citation
Academic tradition requires you to cite works you base your article on.
When using programs that use GNU Parallel to process data for publication
please cite:

@article{Tange2011a,
  title = {GNU Parallel - The Command-Line Power Tool},
  author = {O. Tange},
  address = {Frederiksberg, Denmark},
  journal = {;login: The USENIX Magazine},
  month = {Feb},
  number = {1},
  volume = {36},
  url = {http://www.gnu.org/s/parallel},
  year = {2011},
  pages = {42-47},
  doi = {http://dx.doi.org/10.5281/zenodo.16303}
}

(Feel free to use \nocite{Tange2011a})

This helps funding further development; AND IT WON'T COST YOU A CENT.
If you pay 10000 EUR you should feel free to use GNU Parallel without citing.

If you send a copy of your published article to tange@gnu.org, it will be
mentioned in the release notes of next version of GNU Parallel.
\end{verbatim}
\end{frame}

\begin{frame}[fragile]
\big{``Shut up, parallel!''}
\begin{verbatim}
~ :) parallel --citation

[...]

Type: 'will cite' and press enter.
> will cite

Thank you for your support. It is much appreciated.
The citation notice is now silenced.
\end{verbatim}
\end{frame}

\begin{frame}
\big{The mental model}

Use \texttt{parallel}'s template language and operators to build a list of commands to run.
\end{frame}

\begin{frame}[fragile]
\big{::: and \{\}}
\begin{description}
    \item[:::] Pass command-line args to template.
    \item[\{\}] Replace with each item in the list.
\end{description}

\begin{verbatim}
~ :) parallel {} ::: date uname 'uptime -s'
Tue Oct 22 13:54:53 EDT 2019
Linux
2019-09-30 13:06:13

~ :) parallel 'du -h {}' ::: *.txt
4.0K	chars.txt
4.0K	demenage.txt
16K	UTF-8-demo.txt
\end{verbatim}
\end{frame}

\begin{frame}[fragile]
\big{Mulitple invocations of :::}
Creates a permutation of the elements
\begin{verbatim}
~ :) parallel 'echo {}' ::: a b ::: 1 2 ::: x y
a 1 x
a 1 y
a 2 x
a 2 y
b 1 x
b 1 y
b 2 x
b 2 y
\end{verbatim}
\end{frame}

\begin{frame}[fragile]
\big{:::+}
Similar to ::: but works pair-wise
\begin{verbatim}
~ :) parallel 'echo {}' ::: a b :::+ 1 2 :::+ x y
a 1 x
b 2 y
\end{verbatim}
\end{frame}

\begin{frame}[fragile]
\big{::: and :::+}
The operators ::: and :::+ can be mixed and matched
\begin{verbatim}
~ :) parallel 'echo {}' ::: a b ::: 1 2 :::+ x y
a 1 x
a 2 y
b 1 x
b 2 y
\end{verbatim}
\end{frame}

\begin{frame}[fragile]
\big{Positional placeholders}
The placeholder \texttt{\{n\}} can be used to select the nth element in the list item.
\begin{verbatim}
~ :) parallel 'echo {3} {1}' ::: a b ::: 1 2 :::+ x y
x a
y a
x b
y b
\end{verbatim}
\end{frame}

\end{document}
