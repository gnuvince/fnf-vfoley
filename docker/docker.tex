\documentclass{beamer}

\usefonttheme{serif}
\usepackage{ebgaramond}

\usepackage[utf8]{inputenc}
\usepackage[english]{babel}
\usepackage{listings}
\usepackage{graphicx}
\usepackage{xcolor}
\usepackage{tcolorbox}
\tcbuselibrary{listings}

\definecolor{MyYellow}{rgb}{1.0, 1.0, 0.8}

\setbeamertemplate{navigation symbols}{}
\setbeamertemplate{footline}[frame number]{}
% \setbeamertemplate{footline}{\hfill\insertframenumber~\vrule~\inserttotalframenumber}


\lstdefinestyle{mystyle}{
  stepnumber=1,
  breaklines=true,
  basicstyle=\ttfamily\scriptsize,
  commentstyle=\color{gray},
  showstringspaces=false,
  keepspaces=true,
  escapeinside=\^\^
}

\newtcblisting{mylisting}{
      arc=0pt,
      top=0mm,
      bottom=0mm,
      left=0mm,
      right=0mm,
      boxrule=0pt,
      colback=MyYellow,
      listing only,
      listing options={style=mystyle}
}

\renewcommand\big[1]{
  \begin{center}
    \Large{#1}
  \end{center}
}

\begin{document}

\begin{frame}[fragile]
  \begin{center}Debian, Ubuntu, CentOS\end{center}
  \begin{mylisting}
    > curl https://get.docker.com | sudo sh
    > sudo usermod -aG docker $USER
    [log out / log in]
  \end{mylisting}

  \begin{center}Arch Linux\end{center}
  \begin{mylisting}
    > sudo pacman -S docker
    > sudo usermod -aG docker $USER
    [log out / log in]
    [Start Docker service]
  \end{mylisting}

  \begin{center}macOS\end{center}
  \begin{mylisting}
    > brew cask install docker
  \end{mylisting}
\end{frame}

\begin{frame}[fragile]
  \begin{mylisting}
> docker version

Client:
 Version:       18.02.0-ce
 API version:   1.36
 Go version:    go1.9.3
 Git commit:    fc4de44
 Built:         Wed Feb  7 21:16:24 2018
 OS/Arch:       linux/amd64
 Experimental:  false
 Orchestrator:  swarm

Server:
 Engine:
  Version:      18.02.0-ce
  API version:  1.36 (minimum version 1.12)
  Go version:   go1.9.3
  Git commit:   fc4de44
  Built:        Wed Feb  7 21:14:54 2018
  OS/Arch:      linux/amd64
  Experimental: false
\end{mylisting}
(If the \emph{Server} section is missing, you don't have permissions: add
yourself to the \texttt{docker} group and log back in.)
\end{frame}

\begin{frame}
  \big{Containers?}

  \begin{itemize}
    \item Build, package, and run an application in a \emph{segregated environment}
    \item Namespaces provide own filesystem, process list, network stack, etc.
    \item Cgroups to limit CPU, memory, etc.
    \item More lightweight than VMs (e.g., VMWare)
  \end{itemize}
\end{frame}

\begin{frame}
  \big{Docker?}

  \centering Containers made easy
\end{frame}

\begin{frame}
  \centering\Huge{Using a container}
\end{frame}

\begin{frame}[fragile]
  \big{Running Alpine Linux}
  \begin{mylisting}
    > docker container run -i -t alpine:latest sh

    / # echo Hello inside Docker
    Hello inside Docker
    / # touch FnF
  \end{mylisting}
  \begin{itemize}
    \item \texttt{docker container run}: create a new container and run it;
    \item \texttt{-i -t}: attach terminal to container;
    \item \texttt{alpine:latest}: the image name; typically, \emph{app:version};
    \item \texttt{sh}: inside the container, run a shell.
  \end{itemize}
  (The image is downloaded from \url{hub.docker.com})
\end{frame}

\begin{frame}[fragile]
  \big{List running containers}
  \begin{mylisting}
    > docker container ls

CONTAINER ID        IMAGE               COMMAND
879f1a834085        alpine:latest       "sh"

CREATED              STATUS             NAMES
About a minute ago   Up About a minute  zealous_kalam
\end{mylisting}

\begin{itemize}
  \item \emph{Container Id} and \emph{Name} can be used to refer to container;
  \item A custom name can be given with the \texttt{--name} option.
  \end{itemize}
\end{frame}

\begin{frame}[fragile]
  \big{See processes running in a container}
  \begin{mylisting}
    > docker container top zealous_kalam

UID                 PID            PPID           C
root                23777          23764          0

STIME               TTY            TIME           CMD
21:41               pts/0          00:00:00       sh
  \end{mylisting}
  \begin{itemize}
    \item PID, PPID: PIDs on the \emph{host} machine
  \end{itemize}
\end{frame}

\begin{frame}[fragile]
  \big{View what happened in a container}
  \begin{mylisting}
    > docker container logs zealous_kalam

    / # echo Hello inside Docker
    Hello inside Docker
    / # touch FnF
  \end{mylisting}
\end{frame}

\begin{frame}[fragile]
  \big{Stop a container}
  \begin{mylisting}
    / # exit
  \end{mylisting}
  or
  \begin{mylisting}
    > docker container stop zealous_kalam
  \end{mylisting}
\end{frame}

\begin{frame}[fragile]
  \big{List all containers}
  \begin{mylisting}
> docker container ls -a

CONTAINER ID    IMAGE            COMMAND   CREATED
879f1a834085    alpine:latest    "sh"      13 minutes ago

STATUS                                 NAMES
Exited (137) About a minute ago        zealous_kalam

  \end{mylisting}
\end{frame}

\begin{frame}[fragile]
  \big{Restart an existing container}
  \begin{mylisting}
> docker container start -ai zealous_kalam

/ # ls
FnF    dev    home   media  proc   run    srv    tmp    var
bin    etc    lib    mnt    root   sbin   sys    usr
  \end{mylisting}
\end{frame}

\begin{frame}[fragile]
  \big{Delete a container}
  \begin{mylisting}
    > docker container rm zealous_kalam
  \end{mylisting}
\end{frame}

\begin{frame}[fragile]
  \big{Image vs. Container?}

  Analogy with OO programming:
  \begin{itemize}
    \item Image $\approx$ class
    \item Container $\approx$ instance
  \end{itemize}
\end{frame}

\begin{frame}[fragile]
  \big{Image vs. Container?}
  \begin{mylisting}
> docker container run -d -p 1111:80 nginx:latest
> docker container run -d -p 2222:80 nginx:latest
^\pause^
~$ docker container ls
CONTAINER ID    IMAGE          ...    PORTS
889f13f72315    nginx:latest   ...    0.0.0.0:2222->80/tcp
000209eb0559    nginx:latest   ...    0.0.0.0:1111->80/tcp
^\pause^
> curl -s localhost:1111 | grep Welcome
<title>Welcome to nginx!</title>
<h1>Welcome to nginx!</h1>
^\pause^
> curl -s localhost:2222 | grep Welcome
<title>Welcome to nginx!</title>
<h1>Welcome to nginx!</h1>
  \end{mylisting}
\end{frame}

\begin{frame}
  \big{Images}
\end{frame}

\begin{frame}[fragile]
  \big{View local images}
  \begin{mylisting}
    > docker image ls
REPOSITORY    TAG     ...  CREATED             SIZE
nginx         latest  ...  6 days ago          109MB
alpine        latest  ...  2 months ago        4.15MB
  \end{mylisting}
\end{frame}

\begin{frame}[fragile]
  \big{Pull an image}

  \begin{mylisting}
    > docker image pull debian:9.3

    > docker image ls
REPOSITORY    TAG     ...   CREATED             SIZE
nginx         latest  ...   6 days ago          109MB
debian        9.3     ...   4 weeks ago         100MB
alpine        latest  ...   2 months ago        4.15MB
  \end{mylisting}

  Download an image from DockerHub, but don't create a container.
\end{frame}

\begin{frame}[fragile]
  \big{Remove an image}

  \begin{mylisting}
    > docker image rm debian:9.3
Untagged: debian:9.3
Untagged: debian@sha256:4fcd8c0b6f5e3bd44a16def331
Deleted: sha256:1b3ec9d977fb413627aca6244b
Deleted: sha256:8568818b1f7f534832b393c531

    > docker image ls
REPOSITORY    TAG     ...  CREATED             SIZE
nginx         latest  ...  6 days ago          109MB
alpine        latest  ...  2 months ago        4.15MB
  \end{mylisting}
\end{frame}

\begin{frame}
  \big{Let's create our own image!}
\end{frame}

\begin{frame}[fragile]
  \big{hello.py}
  \begin{mylisting}
    print("Hello, FnF!")
  \end{mylisting}

  \big{Dockerfile}
  \begin{mylisting}
    # Use python:3.6 as our base image
    FROM python:3.6

    # Copy `hello.py` (on host) to
    # `/app/hello.py` (on container)
    COPY hello.py /app/hello.py

    # Invoke `hello.py` when launching container
    ENTRYPOINT /usr/bin/python /app/hello.py
  \end{mylisting}
\end{frame}

\begin{frame}[fragile]
  \big{Build the image and run a container}

  \begin{mylisting}
> docker image build -t fnf:1 .
Sending build context to Docker daemon  3.072kB
<elided>
Successfully built ef953d6e46fb
Successfully tagged fnf:1

> docker container run fnf:1
Hello, FnF!
  \end{mylisting}
\end{frame}

\end{document}
