\documentclass{beamer}

\usefonttheme{serif}
\usepackage{charter}

\usepackage[utf8]{inputenc}
\usepackage[english]{babel}
\usepackage{listings}
\usepackage{graphicx}
 \usepackage{epstopdf}

\setbeamertemplate{navigation symbols}{}
\setbeamertemplate{footline}[frame number]{}
%\setbeamertemplate{footline}{\hfill\insertframenumber~\vrule~\inserttotalframenumber}
\lstset{
  stepnumber=1,
  breaklines=true,
  basicstyle=\ttfamily\scriptsize,
  numberstyle=\tiny,
  commentstyle=\color{gray},
  showstringspaces=false,
  keepspaces=true,
  escapeinside=\#\#
}

\renewcommand\big[1]{
  \begin{center}
    \Large{#1}
  \end{center}
}

\begin{document}

\begin{frame}
  \centering\Huge{awk}
\end{frame}

\begin{frame}
  \big{demo.txt}
  \lstinputlisting{demo.txt}
\end{frame}

\begin{frame}
  \centering\Huge{Principle \#1}
  \big{Patterns and actions}
\end{frame}

\begin{frame}[fragile]
  \big{Patterns and actions}

  \begin{itemize}
    \item A awk program is a list of patterns and actions to execute;
    \item Patterns are tried in order for every line of input;
    \item More than one pattern can match.
  \end{itemize}

  \begin{lstlisting}
    pattern_1 { actions_1 }
    pattern_2 { actions_2 }
    ...
    pattern_k { actions_k }
  \end{lstlisting}
\end{frame}

\begin{frame}[fragile]
  \big{Execution model}

  \begin{lstlisting}
    for each $line {
      for each $pattern, $action {
        if $line matches $pattern {
          execute $action
        }
      }
    }
  \end{lstlisting}
\end{frame}

\begin{frame}[fragile]
  \big{Examples}
  \begin{lstlisting}
    /erlang/                     { erl_projects++ }
    length($3) < 5               { print $1, $3 }
    $2 == "backend" && $3 == "c" { print $0 }
  \end{lstlisting}
\end{frame}

\begin{frame}[fragile]
  \big{Demo}
  \begin{lstlisting}
    > awk '/erlang/ { erl_projects++ }' demo.txt
    #\pause#

    > awk 'length($3) < 5 { print $1, $3 }' demo.txt
    rtb-trader ruby
    iplist-service c
    #\pause#

    > awk '$2=="backend" && $3=="c" { print $0 }' demo.txt
    iplist-service  backend     c
  \end{lstlisting}
\end{frame}

\begin{frame}[fragile]
  \big{Two special patterns}

  \begin{lstlisting}
    BEGIN { /* run before any line is processed */ }
    END   { /* run after all lines are processed */ }
  \end{lstlisting}

  \begin{center}Useful for initialization and reporting.\end{center}
\end{frame}

\begin{frame}[fragile]
  \big{Example}
  \begin{lstlisting}
    > awk '/erlang/ { p++ } END { print p }' demo.txt
    2
  \end{lstlisting}
\end{frame}

\begin{frame}
  \centering\Huge{Principle \#2}
  \big{Abbreviate common use cases}
\end{frame}

\begin{frame}[fragile]
  \big{No pattern}

  If no pattern is specified, all lines match.

  \begin{lstlisting}
    > awk '{print $1}' demo.txt
    rtb-gateway
    rtb-trader
    iplist-service
    identifyd
    rtb-delivery
  \end{lstlisting}
\end{frame}

\begin{frame}[fragile]
  \big{No action}

  If no action is specified, the whole line is printed.

  \begin{lstlisting}
    > awk '/data|backend/' demo.txt
    rtb-gateway     backend     erlang
    iplist-service  backend     c
    identifyd       data        scala
    rtb-delivery    backend     erlang
  \end{lstlisting}
\end{frame}

\begin{frame}[fragile]
  \big{Print without arguments}

  \texttt{print} without arguments prints the whole line.

  \begin{lstlisting}
    > awk '$2 == "data" || $3 == "c" { print }' demo.txt
    iplist-service  backend     c
    identifyd       data        scala
  \end{lstlisting}
\end{frame}

\end{document}
