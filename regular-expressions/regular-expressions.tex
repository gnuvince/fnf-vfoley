\documentclass{beamer}

\usefonttheme{serif}
\usepackage{charter}

\usepackage[utf8]{inputenc}
\usepackage[english]{babel}
\usepackage{listings}
\usepackage{graphicx}
\usepackage{epstopdf}
\usepackage{color}

\setbeamertemplate{navigation symbols}{}
\setbeamertemplate{footline}[frame number]{}
%\setbeamertemplate{footline}{\hfill\insertframenumber~\vrule~\inserttotalframenumber}
\lstset{
  stepnumber=1,
  breaklines=true,
  basicstyle=\ttfamily\scriptsize,
  numberstyle=\tiny,
  commentstyle=\color{gray},
  showstringspaces=false,
  keepspaces=true,
  escapeinside=\#\#
}

\renewcommand\big[1]{
  \begin{center}
    \Large{#1}
  \end{center}
}

\begin{document}

\begin{frame}
  \centering\Huge{The Fundamentals of Regular Expressions}
\end{frame}

\begin{frame}
  \big{The Building Blocks}
  \begin{center}
    \begin{tabular}{ll}
      $\epsilon$ & Match the empty string \\
      $c$ & Match the character $c$ \\
      $R_1 R_2$ & Match $R_1$ followed by $R_2$ \\
      $R_1 \vert R_2$ & Match $R_1$ or $R_2$ \\
      $R*$ & Match $R$ zero or more times
    \end{tabular}
  \end{center}
\end{frame}

\begin{frame}
  \big{Examples}
  \begin{itemize}
    \item $w$ : match the character $w$
    \item $y \vert n$ : match the character $y$ or the character $n$
    \item $bar$ : match the character $b$, followed by $a$, followed by $r$
    \item $yo!*$ : match $yo$, followed by any number of bangs
    \item $(yo!)*$ : match any number of $yo!$
  \end{itemize}
\end{frame}

\begin{frame}
  \big{The Shortcuts}
  \begin{center}
    \begin{tabular}{lll}
      $R?$ & Match $R$ 0 or 1 times & $R \vert \epsilon$ \\
      $R+$ & Match $R$ 1 or more times & $RR*$ \\
      $R\{n\}$ & Match $R$ exactly $n$ times & $RRR...R$ ($n$ times) \\
      $[xyz]$ & Match the character $x$, $y$, or $z$ & $x \vert y \vert z$ \\
      $[a-z]$ & Match the character $a$, $b$, ..., or $z$ & $a \vert b \vert ... \vert z$
    \end{tabular}
  \end{center}
\end{frame}

\begin{frame}
  \big{How matching works}

  Running example: matching integers

  \[
    (+ \vert -)?[0-9]+
  \]
  \pause
  \[
    (+ \vert - \vert \epsilon)~(0 \vert 1 \vert ... \vert 9)~(0 \vert 1 \vert ... \vert 9)*
  \]
\end{frame}


\begin{frame}
  \big{How matching works}

  \begin{align*}
    & \textcolor{red}{(+ \vert - \vert \epsilon)}~(0 \vert 1 \vert ... \vert 9)~(0 \vert 1 \vert ... \vert 9)* \\
    & \textcolor{red}{-}~1~7~9
  \end{align*}
\end{frame}

\begin{frame}
  \big{How matching works}

  \begin{align*}
    & (+ \vert - \vert \epsilon)~\textcolor{red}{(0 \vert 1 \vert ... \vert 9)}~(0 \vert 1 \vert ... \vert 9)* \\
    & -~\textcolor{red}{1}~7~9
  \end{align*}
\end{frame}

\begin{frame}
  \big{How matching works}

  \begin{align*}
    & (+ \vert - \vert \epsilon)~(0 \vert 1 \vert ... \vert 9)~\textcolor{red}{(0 \vert 1 \vert ... \vert 9)*} \\
    & -~1~\textcolor{red}{7}~9
  \end{align*}
\end{frame}

\begin{frame}
  \big{How matching works}

  \begin{align*}
    & (+ \vert - \vert \epsilon)~(0 \vert 1 \vert ... \vert 9)~\textcolor{red}{(0 \vert 1 \vert ... \vert 9)*} \\
    & -~1~7~\textcolor{red}{9}
  \end{align*}
\end{frame}

\begin{frame}
  \big{How matching works}

  \begin{align*}
    & (+ \vert - \vert \epsilon)~(0 \vert 1 \vert ... \vert 9)~\textcolor{red}{(0 \vert 1 \vert ... \vert 9)*} \\
    & -~1~7~9
  \end{align*}

  \centering\Huge{:)}
\end{frame}

\begin{frame}
  \big{How matching works}

  \begin{align*}
    & \textcolor{red}{(+ \vert - \vert \epsilon)}~(0 \vert 1 \vert ... \vert 9)~(0 \vert 1 \vert ... \vert 9)* \\
    & \textcolor{red}{3}
  \end{align*}
\end{frame}


\begin{frame}
  \big{How matching works}

  \begin{align*}
    & (+ \vert - \vert \epsilon)~\textcolor{red}{(0 \vert 1 \vert ... \vert 9)}~(0 \vert 1 \vert ... \vert 9)*\\
    & \textcolor{red}{3}
  \end{align*}
\end{frame}


\begin{frame}
  \big{How matching works}

  \begin{align*}
    & (+ \vert - \vert \epsilon)~(0 \vert 1 \vert ... \vert 9)~\textcolor{red}{(0 \vert 1 \vert ... \vert 9)*} \\
    & 3
  \end{align*}
  \centering\Huge{:)}
\end{frame}

\begin{frame}
  \big{How matching works}

  \begin{align*}
    & \textcolor{red}{(+ \vert - \vert \epsilon)}~(0 \vert 1 \vert ... \vert 9)~(0 \vert 1 \vert ... \vert 9)* \\
    & \textcolor{red}{+}~{z}
  \end{align*}
\end{frame}

\begin{frame}
  \big{How matching works}

  \begin{align*}
    & (+ \vert - \vert \epsilon)~\textcolor{red}{(0 \vert 1 \vert ... \vert 9)}~(0 \vert 1 \vert ... \vert 9)* \\
    & {+}~\textcolor{red}{z}
  \end{align*}

  \centering\Huge{:(}
\end{frame}


\begin{frame}
  \big{How matching works}

  \begin{align*}
    & \textcolor{red}{(+ \vert - \vert \epsilon)}~(0 \vert 1 \vert ... \vert 9)~(0 \vert 1 \vert ... \vert 9)* \\
    & \textcolor{red}{-}
  \end{align*}
\end{frame}

\begin{frame}
  \big{How matching works}

  \begin{align*}
    & (+ \vert - \vert \epsilon)~\textcolor{red}{(0 \vert 1 \vert ... \vert 9)}~(0 \vert 1 \vert ... \vert 9)* \\
    & {-}
  \end{align*}

  \centering\Huge{:(}
\end{frame}

\begin{frame}
  \big{The Building Blocks}
  \begin{center}
    \begin{tabular}{ll}
      $\epsilon$ & Match the empty string \\
      $c$ & Match the character $c$ \\
      $R_1 R_2$ & Match $R_1$ followed by $R_2$ \\
      $R_1 \vert R_2$ & Match $R_1$ or $R_2$ \\
      $R*$ & Match $R$ zero or more times
    \end{tabular}
  \end{center}
\end{frame}

\begin{frame}
  \big{In practice}
  \begin{itemize}
    \item Many more shortcuts:
    \begin{itemize}
      \item $.$ for any character (except, sometimes, newline)
      \item $\backslash{}d$ for digit
      \item $\backslash{}w$ for whitespace
      \item etc.
    \end{itemize}
    \item Rules for escaping:
    \begin{itemize}
        \item In Perl, $($ and $)$ are for grouping; use $\backslash{}($ and $\backslash{})$ to match parentheses (make sure to backslash the backslash!)
      \item In Emacs, $($ and $)$ match themselves; use backslashes for grouping
    \end{itemize}
    \item Greedy vs. lazy
  \end{itemize}
\end{frame}

\begin{frame}
  \big{In practice}

  \big{Same basic principles apply}
  \pause
  \centering{Until you get into back-references...}
\end{frame}

\begin{frame}
  \centering\Huge{Fin}
\end{frame}

\end{document}
